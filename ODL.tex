\chapter{ODL}
\begin{figure}[h]
    \centering
   \includegraphics[width=1\textwidth]{ODL-Architecture.png}
    \caption{Architettura del controller OpenDayLight}
    \label{fig:ArchODL}
\end{figure}
OpenDaylight \cite{ODL} è un progetto open source che utiliza protocolli aperti al fine di fornire controlli centralizzati e gestire il monitoring della rete.
Fa parte della fondazione LF Networking \cite{LFN} che si occupa di stanziare e coordinare il supporto a progetti open source coinvolti nello spostamento o nella comunicazione di dati su una rete.
Ciò semplifica il coinvolgimento dei membri e aumenta la collaborazione tra i progetti e gli organismi di standardizzazione.
%Lo scopo principale è far crescere l'ecosistema di ODL per facilitare la collaborazione tra sviluppatori, utenti finali e aziende associate a LFV per produrre tecnologie più pertinenti e affidabili.
\\ODL è un framework scritto in Java che consente di soddisfare esigenze specifiche dell'utente e fornire quindi alta flessibilità.
La caratteristica principale è la sua architettura costituita da microservizi che un utente può decidere se abilitare o meno. 
Di default sono tutti disabilitati così da permettere una totale personalizzazione.
Per la gestione dei moduli a runtime e l'installazione di funzionalità Karaf da implementare nel software di ODL viene utilizzato Apache Karaf \cite{Apache}.
Attraverso il framework MD-SAL (Model-Driven Service Abstraction Layer) permette agli sviluppatori di creare nuove features sotto forma di servizi e protocolli collegati tra loro.
\\Per far fronte ai problemi di scalabilità, disponibilità e persistenza dei dati si possono avere più instanze di ODL distribuite su macchine differenti che cooperano tra loro attraverso il meccanismo dei cluster.
La comunicazione con le applicazioni si realizza grazie a diverse API northbound quali REST o NETCONF, mentre per interagire 
con i dispositivi di rete offre supporto a diversi protocolli southbound come OpenFlow o NETCONF.


\section {Installazione}
Il controller viene eseguito all'interno di una Java Virtual Machine (JVM), quindi è necessario verificare quali versioni di Java supporta la distribuzione che si decide di installare.
La versione stabile più recente di ODL è compatibile con le versioni di Java superiori alla 17, mentre per le versioni più datate quest'ultime non vanno bene. 
Per l'installazione si deve scaricare la distribuzione desiderata che si trova sul loro sito ufficiale nella pagina di download del software \cite{InstallODL}.
Successivamente è necessario fare l'unzip del file, navigare nella cartella e eseguire il seguente comando da terminale per avviare il controller.
\begin{lstlisting}[language=CLI]
./bin/karaf
\end{lstlisting}
La versione di ODL dell'immagine \ref{fig:installazione} è Potassium.
\begin{figure}[h]
    \centering
   \includegraphics[width=1\textwidth]{odl/Potassium.png}
    \caption{Installazione ODL}
    \label{fig:installazione}
\end{figure}
\\Successivamente, si può trovare una lista completa delle feature disponibili eseguendo il seguente comando.
\begin{lstlisting}[language=CLI]
feature:list
\end{lstlisting}
\begin{figure}[h]
    \centering
   \includegraphics[width=1\textwidth]{odl/feature.png}
    \caption{Alcune features disponibili}
    \label{fig:feature}
\end{figure}
Per installarle invece 
\begin{lstlisting}[language=CLI]
feature:install <feature1> <feature2>..
\end{lstlisting}
