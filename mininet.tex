\chapter{Mininet}
Mininet \cite{mininet} è un sistema open source di orchestrazione per l'emulazione di reti in grado di creare una realistica rete virtuale.
Presenta delle API e un interprete Python che consentono di definire degli script attraverso i quali è possibile creare delle topologie di rete.
Ciò è possibile anche tramite Command line interface (CLI)  e possono essere sia predefinite che personalizzate tramite la possibilità di 
aggiungere e rimuovere switch, router, host, controller e link.
\\Mininet mette a disposizione tre livelli differenti di API:
\begin{itemize}
\item Low-level: consiste nelle classi dei nodi e dei link istanziati individualmente e usati per creare una rete.
\item Mid-level: aggiunge un containter per nodi e link, l'oggetto Mininet, e fornisce metodi per la configurazione di rete.
\item High-level: aggiunge l'astrazione della topologia di rete, la classe Topo. Offre la possibilità di creare modelli di topologia riusabili passandoli al comando mn da linea di comando.
\end{itemize}
Per rendere la rete più realistica e adatta a esperimenti di test si possono configurare i link come up o down e inserire metriche specifiche 
come possono essere quelle di banda, ritardo, perdita o massima lunghezza della coda di recezione.
Gli host su Mininet condividono il filesystem root del server sottostante. 
Ciò significa che non è necessario copiare dati tra gli host presenti.
Un effetto collaterale di questa condivisione si può ritrovare quando un programma richiede specifiche di configurazione in quanto si dovrà creare un differente 
file di configurazione per ogni host presente e specificarlo come opzione di avvio del programma. 
%Inoltre ci possono essere collisioni tra file se si prova a creare lo stesso file nella stessa directory di più hosts.
%Mininet mette a disposizione una GUI (miniedit) utile per visualizzare lo stato della rete durante gli esperimenti svolti.
Mininet è stato progettato per essere facilmente integrabile con altri software e sistemi di rete.
Consente di connettere un controller remoto agli switch, indipendentemente dal PC su cui è installato, in modo da fornire un ambiente adatto allo sviluppo e al test.
E' possibile quindi collegare un controller SDN che supporta OpenFlow.

