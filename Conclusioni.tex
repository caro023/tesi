\chapter{Conclusioni}
All'interno di questo elaborato sono state inizialmente descritte le motivazioni che stanno spingendo la ricerca di 
nuove tecnologie per superare le limitazioni dell'infrastruttura di rete tradizionale.
Abbiamo presentato il paradigma SDN come uno dei più promettenti per conseguire tale obiettivo e 
sono state discusse le implementazioni più popolari per i controller SDN. 
Ci siamo concentrati sull'architettura e sulle funzionalità di TeraFlow, un controller SDN emergente progettato per le reti B5G, e abbiamo analizzato i suoi meccanismi di creazione e gestione dei servizi di rete.
La sperimentazione si è focalizzata sulla creazione di servizi di connettività tra due end points, 
sull'introduzione di politiche di rete specifiche, e sulla verifica della risposta del sistema a condizioni introdotte artificialmente.
I test sono stati eseguiti utilizzando l'emulatore di rete Mininet.
Inizialmente, è stato condotto un esperimento di base per verificare la connettività tra i nodi e l'applicazione di una politica di routing. 
Successivamente, la topologia è stata estesa per simulare una rete più complessa
e sono state modificate le metriche di monitoraggio e relative alla politica per introdurre 
i vincoli di latenza, packet loss e capacità.
La valutazione finale è stata effettuata su prove di raggiungibilità e performance che, pur
essendo relativamente semplici, rappresentano un punto di partenza per analisi più complesse ed
esegubili su reti più ampie. 
%\newline In una rete reale, tuttavia, andrebbero svolti ulteriori verifiche e test per garantire che le performance raggiunte negli esperimenti siano replicate e mantenute.
%Questo perché vari fattori, come la congestione della rete, le fluttuazioni del traffico e il comportamento degli utenti, possono influenzare significativamente il funzionamento del sistema.
\newline Durante la sperimentazione sono emerse alcune problematiche legate alla mancanza di implementazione
di specifiche funzioni che hanno rallentato lo sviluppo delle funzionalità desiderate.
In particolare, alcune regole di politica avrebbero potuto essere sostituite con vincoli aggiunti durante la fase di creazione del servizio,
ma l'implementazione non era adeguata. 
Inoltre, è stata necessaria la configurazione completa della macchina virtuale a causa di problemi con quella fornita e preconfigurata per l'Hackfest 3.
Un altro inconveniente riscontrato è stato il dover ripetere più volte gli stessi comandi affinché avessero effetto, soprattutto nelle fasi di definizione e di eliminazione della politica.
\newline Nonostante queste limitazioni, l'approccio sperimentale adottato ha permesso di ottenere risultati significativi, 
evidenziando la capacità del sistema di rispondere a condizioni di rete variabili e di applicare politiche di gestione del traffico in modo efficace.
\newline Per lavori futuri si potrebbe pensare di implementare il servizio tramite interfaccia Web
che facilita la configurazione e la gestione del servizio 
consentendo agli utenti di interagire con il sistema senza dover modificare direttamente i file JSON o utilizzare comandi da terminale.
Attualmente i tipi supportati per la creazione del servizio tramite WebUI sono: ACL\_L2 ,ACL\_IPV4, ACL\_IPV6, L2VPN, L3VPN.
Quindi il tipo di servizio utilizzato in questa sperimetazione non è supportato.
\newline Si potrebbe inoltre sperimentare con topologie più grandi e politiche più complesse per approfondire ulteriormente le capacità e i limiti del sistema. 
Infine, si potrebbero verificare le stesse prove  con altri protocolli e tipi di switch differenti da P4,
ampliando così le possibilità di applicazione e la robustezza delle soluzioni proposte.
%futuri lavori: tramite interfaccia web (ancora non implementato), aggiongere kpisample