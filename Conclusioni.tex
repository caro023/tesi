\chapter{Conclusioni}
All'interno di questo elaborato sono state inizialmente descritte le motivazioni che stanno spingendo la ricerca di 
nuove tecnologie per superare le limitazioni dell'infrastruttura di rete tradizionale.
Abbiamo presentato il paradigma SDN come uno dei più promettenti per conseguire tale obiettivo e 
sono state discusse le implementazioni più popolari per i controller SDN. 
Ci siamo concentrati sull'architettura e sulle funzionalità di TeraFlow, un controller SDN emergente progettato per le reti B5G, e abbiamo analizzato i suoi meccanismi di creazione e gestione dei servizi di rete.
La sperimentazione si è focalizzata sulla creazione di servizi di connettività tra due end points, 
l'introduzione di politiche di rete specifiche, e la verifica della risposta del sistema a condizioni introdotte artificialmente.
I test sono stati resi possibili dall'emulatore di rete Mininet.
La valutazione è stata fatta su prove di raggiungibilità e di performance che, pur
essendo abbastanza semplici, rappresentano un punto di partenza per analisi più complesse ed
esegubili su reti più ampie. 
\\In una rete reale, tuttavia, andrebbero svolti ulteriori accertamenti per garantire le stesse performance.
\\Durante la sperimentazione si sono riscontrate delle problematiche dovute alla mancanza di implementazione
di alcune funzioni che hanno limitato lo sviluppo completo delle funzionalità desiderate.
\\Nonostante queste limitazioni, l'approccio sperimentale adottato ha permesso di ottenere risultati significativi, 
evidenziando la capacità del sistema di rispondere a condizioni di rete variabili e di applicare politiche di gestione del traffico in modo efficace.
\\Per lavori futuri si potrebbe pensare di implementare il servizio tramite interfaccia Web.
Attualmente la creazione dei servizi tramite WebUI non supporta il tipo L2MV utilizzato in questa sperimetazione.
\\Si potrebbe inoltre sperimentare con topologie più grandi e politiche più complesse per approfondire ulteriormente le capacità e i limiti del sistema. 
Infine, si potrebbero verificare le stesse prove  con altri protocolli e tipi di switch differenti da P4,
ampliando così le possibilità di applicazione e la robustezza delle soluzioni proposte.
%futuri lavori: tramite interfaccia web (ancora non implementato), aggiongere kpisample