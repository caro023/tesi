\chapter{Studio e sperimentazione della gestione di policy in Teraflow}
\label{cap:policy}
Lo scopo della tesi è quello di stabilire una connessione, attraverso dispositivi che supportano p4, tra due end points nella rete che rispetti determinate caratteristiche utilizzando il controller SDN Teraflow.
Inizialmente, attraverso il Service Level, è stato introdotto un servizio nella rete sotto forma di intento per stabilire la connessione e il percorso. 
Successivamente, tramite il Management Level, è stata inserita una politica basata sugli eventi. Questa politica permette agli operatori 
di associare a un servizio scelto un service level agreement (SLA) con specifiche condizioni che devono essere rispettate a run time.
%end-to-end
%tutte queste info sono persistite in un database logicamente centralizzato, fisicamente distribuito e scalabile, dato dalla componente di Context
\\Per iniziare il lavoro si è partiti da una demo già esitente per poi apportare delle modifiche successivamente.
\\\textbf{Servizio end-to-end}
\\Come già detto in precedenza il Devide Level sfrutta una South-Bound Interface (SBI) per interagire con i device tramite l'API P4Runtime. 
Inizialmente, il codice p4 compilato viene copiato nel pod SBI per poter inserire le giuste configurazioni nelle tabelle dei dispositivi.
Il passo successivo è registrare i dispositivi e i link al controller SDN per permettere una corretta comunicazione tra di essi.
A questo punto, siamo in grado di richiedere una connessione tra due end points specificando solamente i dispositivi finali tramite un servizio.
La componente di Service, ricevuta la richiesta, si rivolge alla PathComp per calcolare un percorso.
Per eseguire questa operazione la PathComp utilizza delle informazioni della rete che persistono nel database logicamente centralizzato della componente di Context.
\\Una volta ricevuto il percorso, la componente di Service configura i dispositivi lungo il percorso tramite l'SBI.
Per mantenere questo processo agnostico rispetto ai dettagli della tecnologia, la componente di Service 
sfrutta una definizione minima permettendo agli utenti di esprimete cosa vogliono connettere, lasciando che sia il sistema sottostante a decidere come realizzare la connessione reale.
La componente traduce automaticamente questa definizione minimale del servizio in modelli di configurazioni astratte dei dispositivi. 
Queste vengono a loro volta tradotte in regole P4 dal driver del dispositivo P4 della SBI.
\\\textbf{Constraints e action rules}
\\Nella demo non erano presenti ma abbiamo introdotto dei contraisnt o vincoli che dovevano essere rispettati dal servizio.
%spiegare cosa sono i constraits, guarda i deliverable
%dire che non sono implementati
%run-time-->loss+latenza
\\\textbf{Politica}
\\Una parte fondamentale nei sistemi moderni è la gestione a run-time del servizio stabilito.
A tale scopo, si sfrutta la componente di Monitoring, che permette di associare il monitoraggio delle metriche nel proprio database,
con condizioni che devono essere rispettate.
Quando queste condizioni non vengono soddisfatte, la componente di Monitoring solleva un allarme che fa scattare l'azione prestabilita.
\\Nella sperimentazione abbiamo introdotto una politica per il servizio stabilito con due condizioni: una per la latenza (che non deve superare i 100ms) e una per la loss (che non deve superare il 20\%).
Queste due condizioni sono legate tra loro tramite un "OR", quindi appena una delle due non è più rispettata, viene invocata l'azione.
In questo caso l'azione consiste nel ricalcolo del percorso.
\\\textbf{Verifica}

%parte di test in mininet
%modifica alla topologia della demo
%probe
