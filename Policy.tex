\chapter{Studio e sperimentazione della gestione di policy in Teraflow}
\label{cap:policy}
%Lo scopo della tesi è quello di stabilire una connessione tra due end points nella rete che rispetti determinate caratteristiche
%a tempo di esecuzione utilizzando il controller SDN Teraflow.
In questo Capitolo verranno esposti i vari esperimenti eseguiti utilizzando il controller SDN TeraFlow per quanto riguarda la gestione degli intenti.
Nella fase iniziale si è seguito l'Hackfest 3 \cite{hackfest} dove è stato fornito un ambiente pre-configurato per testare il controller. 
Il lavoro è iniziato con l'installazione della macchina virtuale \cite{VM} creata appositamente per il congresso.
%Gli esperimenti sono stati svolti all'interno di un'infrastruttura virtualizzata configurata per eseguire sia TeraFlow che Mininet, permettendo di simulare e analizzare vari scenari di rete.
%La macchina virtuale (VM) utilizzata inizialmente per gli esperimenti è stata quella utilizzata durante l'Hackfest 3 \cite{hackfest}, dove è stato fornito un ambiente pre-configurato per testare TeraFlow SDN. 
\\Di seguito vengono riportate le specifiche della VM utilizzata:
\begin{itemize}
    \item IP Address: 10.0.2.X/24 (DHCP)
    \item Gateway: 10.0.2.1
    \item DNS: 8.8.8.8, 8.8.4.4
    \item Creata con VirtualBox 6.1 ma compatibile con versioni successive
    \item Requisiti minimi: 4 vCPU, 6 GB di RAM, 50 GB di spazio sul disco, Virtual Disk Image (VDI)
    \item Connessione di rete: NAT Network con porte esposte 22 per SSH e 80 per HTTP
    \item Sistema operativo senza interfaccia grafica per ridurre il consumo di risorse.
\end{itemize}
La VM ha al suo interno preinstallati MicroK8s con le componenti richieste e Mininet in formato docker.
La versione di TeraFlow utilizzata è la 2.1 con degli adattamenti specifici per l'Hackfest.
\\In seguito è stata configurata una nuova VM con la versione aggiornata di TeraFlow (3.0) a causa di un'incompleta implementazione della componente di Policy che non era in grado di gestire e riconoscere le KPI.
Le caratteristiche richieste sono sostanzialmente simili, con un aumento di memoria RAM a 8 GB e dello spazio su disco a 60 GB.
Il sistema operativo utilizzato è Ubuntu Server 22.04.4 LTS, compatibile anche con la versione successiva 22.04.6 LTS. 
Inoltre è richiesta l'installazione di MicroK8 v1.24.17 con le componenti necessarie, Docker e la versione 3.9.18 di Python.
Per la gestione di mininet è stato integrata la cartella ngsdn-tutorial \cite{ngsdn}.
\\I vari esperimenti si sono svolti seguendo il seguente schema:
attraverso il Service Level è stato introdotto un servizio nella rete sotto forma di intento per stabilire la connessione e il percorso. 
Successivamente, tramite il Management Level, è stata inserita una politica basata sugli eventi che ha consentito di associare un Service Level Agreement (SLA) a un servizio specifico,
includendo condizioni da monitorate e da far rispettate durante l'esecuzione.
%modifica file (NodePort..)
%modifica per abilitare più porte 
%end-to-end
%tutte queste info sono persistite in un database logicamente centralizzato, fisicamente distribuito e scalabile, dato dalla componente di Context
\\Per iniziare il lavoro si è partiti da una demo già preesistente, apportando in seguito le modifiche necessarie.
\section{Strumenti per la sperimentazione}
Prima di passare alla sperimentazione verranno descritti due strumenti che sono stati fondamentali per questa fase.
L'uso di Mininet è stato necessario per la mancanza di una rete reale e, i dispositivi emulati sono 
stati degli switch P4.
Questi hanno permesso, attraverso la loro programmazione, di portare a termine i vari esperimenti sui 
servizi e sulle politiche.
\subsection{Mininet}
\label{ch:Mininet}
Mininet \cite{mininet} è un sistema open source per l'emulazione di reti, su un unico ambiente Linux, che permette di emulare un'intera rete su un singolo computer, come rappresentato in Figura \ref{fig:mininet}.
\begin{figure}[h]
    \centering
   \includegraphics[width=1\textwidth]{mininet.png}
    \caption{Rete mininet}
    %https://telcomaglobal.com/p/what-is-mininet  fonte foto mininet
    \label{fig:mininet}
\end{figure}
E' ampiamente utilizzato in ambiti di ricerca e sviluppo per creare reti virtuali in modo realistico e testare nuove applicazioni, o protocolli, in un ambiente controllato prima di implementarli su reti reali.
\\Mininet, a differenza di altri emulatori che utilizzano macchine virtuali per ogni dispositivo, è in grado di gestire un insieme di terminali di rete
utilizzando la virtualizzazione leggera e consentendo di avviare numerosi host e switch (fino a 4096) \cite{MininetOv}.
Questo è reso possibile da tecnologie implementate nel kernel, come i network namespaces, %KERNEL LINUX
che permettono di creare istanze separate di interfacce di rete, tabelle di routing e tabelle ARP che operano in modo indipendente \cite{tesiMininet}. 
%Rispetto ad altri emulatori presenti in circolazione che emulano ogni dispositivo su una macchina virtuale, Mininet offre una serie di vantaggi. 
%Inanzitutto, permette l'avviamento rapido di una rete, la capacità di eseguire test e programmi con tipologie ampie e personalizzate. 
%Inoltre consente di personalizzare l'inoltro dei pacchetti per testare diverse funzionalità con la possibilità di condividere e replicare il codice.
\\Mininet offre delle API e un interprete Python, o un interfaccia a riga di comando (CLI), che consentono di definire e gestire facilmente le topologie di rete.
In entrambi i casi, si possono scegliere topologie predefinite o personalizzarne di nuove aggiungendo e rimuovendo switch, router, host, controller e link.
Mininet consente anche di configurare regole per l'inoltro dei pacchetti per testare diverse funzionalità, come NAT (Network Address Translation) o ECMP (Equal-Cost Multi-Path), facilitando la replica degli scenari di test in ambienti con caratteristiche differenti.
%switch, router e collegamenti all'interno di un singolo ambiente Linux,
\\Mininet mette a disposizione tre livelli differenti di API \cite{introMin}:
\begin{itemize}
\item \textbf{Low-level}: consiste nelle classi dei nodi e dei link istanziati individualmente e usati per creare una rete.
\item \textbf{Mid-level}: aggiunge un containter per nodi e link, l'oggetto Mininet, e fornisce metodi per la configurazione di rete.
\item \textbf{High-level}: aggiunge l'astrazione della topologia di rete, la classe Topo. Offre la possibilità di creare modelli di topologia riusabili passandoli al comando mn da linea di comando.
\end{itemize}
Si possono anche impostare i link come up o down e inserire metriche specifiche 
come quelle di banda, di ritardo, di perdita o di massima lunghezza della coda di recezione per rendere la rete più realistica e adatta a esperimenti.
%\\Gli host su Mininet condividono il filesystem root del server sottostante. 
%Ciò significa che non è necessario trasferire file tra gli host virtuali perché tutti accedono agli stessi file direttamente.
\\Tuttavia la condivisione delle risorse del sistema può creare problemi.
Inanzitutto queste dovranno essere bilanciate tra tutti gli host della rete,
inoltre, se un programma ha bisogno di file di configurazione specifici per ogni host è necessario crearne uno separato per ogni differente specifica e indicarlo quando si avvia il programma.
%Inoltre ci possono essere collisioni tra file se si prova a creare lo stesso file nella stessa directory di più hosts.
%Mininet mette a disposizione una GUI (miniedit) utile per visualizzare lo stato della rete durante gli esperimenti svolti.
\\Mininet è stato progettato per essere facilmente integrabile con altri software e strumenti di rete,
infatti anche se è fornito un controller di default, consente di connetterne uno remoto agli switch, indipendentemente dal PC su cui è installato.
Questa funzionalià ci ha permesso di poter effettuare la simulazione.
\subsubsection{Alcuni comandi fondamentali}
\textbf{Linea di comando}
\\Inanzitutto è fondamentale creare una topologia di rete con il seguente comando\cite{walkmin}:
\\\textit{\$ sudo mn}
\\Di default è stata inizializzata la topologia minimale (--topo=minimal) che consiste in uno switch connesso a due host e un controller OpenFlow.
All'interno di Mininet si possono trovare altre topologie disponibili e visualizzabili con il comando \\\textit{\$sudo mn -h} \\che si possono specificare tramite l'opzione $--topo$.
%scrivi le varie topologie
%--topo single, 3 uno switch con 3 host
\\Per avviare la topologia esistono diverse opzioni da poter applicare.Ad esempio, l'opzione $--controller$ seguito dall'indirizzo IP specifica il controller al quale gli switch dovranno collegarsi al posto 
del predefinito offerto da Mininet.
%riguarda
\\Una volta creata la topologia per avere informazioni su di essa esistono diversi comandi:
%mettere output/immagini
\begin{itemize}
    \item \textit{ nodes}: per visualizzare i nodi presenti.
    \item \textit{ net}: per visualizzare i nodi e i link presenti.
    \item \textit{ dump}: per visualizzare tutte le informazioni di dump dei nodi.
    \item \textit{h1 ifconfig}: per visualizzare le interfacce del nodo h1.
\end{itemize}
Alcuni comandi per interagire con la rete e fare dei test minimali sono:
\begin{itemize}
    \item \textit{ h1 ping -c 1 h2 }: verifica il corretto funzionamento del percorso tra h1 e h2.
    \item \textit{ pingall}: esegue il ping tra tutti gli host connessi alla rete.
    \item \textit{ iperf}: esegue un test di banda tra 2 degli host della rete.
    \item \textit{xterm h1}: permette di avviare il terminale relativo al nodo h1.
    \item \textit{exit}: esce dalla rete.
\end{itemize}
Per manipolare le metriche relative ai link invece vengono messi a disposizione i seguenti comandi:
\begin{itemize}
    \item \textit{ link s1 h1 down}: disabilita un link, in questo caso quello tra lo switch s1 e l'host h1.
    \item \textit{ link s1 h1 up}: attiva un link, in questo caso quello tra lo switch s1 e l'host h1.
    \item \textit{s2 tc qdisc add dev s2-eth2 root netem loss 50\% }: aggiunge una packet loss del 50\% sulla porta eth2 dello switch s2.
    \item \textit{s2 tc qdisc add dev s2-eth2 root netem delay 200ms}: aggiunge un ritardo di 200ms sulla porta eth2 dello switch s2.
    \item \textit{s2 tc qdisc del dev s2-eth2 root netem loss 50\% }: elimina una packet loss del 50\% sulla porta eth2 dello switch s2.
    \item \textit{s2 tc qdisc del dev s2-eth2 root netem delay 200ms}: elimina un ritardo di 200ms sulla porta eth2 dello switch s2.
\end{itemize} 
\textbf{API Python}
\\Le API Python di Mininet permettono di creare e gestire topologie di rete in modo più flessibile e programmabile. 
Di seguito esponiamo alcune classi e comandi della Mid-level API:
\begin{itemize}
    \item \textit{Mininet}: classe per creare e gestire la rete. Il costruttore prende in input diversi parametri la topologia, gli host, gli switch,i controller, i link e ritorna un oggetto di rete.
    \item \textit{addSwitch()}: aggiunge uno switch alla topologia.
    \item \textit{addHost()}: aggiunge un host alla topologia.
    \item \textit{addLink()}: aggiunge un link alla topologia. Si possono specificare paramentri come la banda espressa in Mbit (bw=10 ), il ritardo (delay='5ms'), massima dimensione della coda espressa in numero di pacchetti (max\_queue\_size=1000), la loss espressa in percentuale (loss=10)
    \item \textit{start}: avvia la rete
    \item \textit{stop}: esce dalla rete
    \item \textit{pingall}: esegue il ping tra tutti gli host connessi alla rete
    \item \textit{h1.cmd('comando da eseguire')}: esegue un comando su h1 da CLI e prende l'output
\end{itemize}
Con le API in Python si può anche estendere il comando \textit{mn} usando l'opzione \textit{--custom} per invocare la topologia ricreata nello script.
\\\textit{sudo mn --your\_script.py --topo your\_topo}

\subsection{P4}
Programming Protocol-indipendent Packet Processor (P4 \cite{p4}), è un linguaggio di programmazione flessibile
che permette di descrivere il comportamento degli elementi di rete, consentendo di personalizzare come i dispositivi elaborano i pacchetti.
%\\P4 è nato quindi con l'obiettivo di definire nuove astrazioni per programmare il piano dati e gestire l'inoltro senza dover andare incontro alle limitazioni riscontrate in precedenza.
\\I principali obiettivi sono:
\begin{itemize}
    \item \textbf{Indipendenza dal protocollo}: P4 non è vincolato a nessun protocollo specifico consentendo anche di definirne di nuovi o modificare quelli esistenti
    \item \textbf{Indipendenza dal target}: Il codice può essere compilato per funzionare su diversi dispositivi, sia hardware che software, rendendolo versatile
    \item \textbf{Riprogrammabilità}: Il comportamento del piano dati può essere aggiornato dinamicamente consentendo di rispondere rapidamente ai cambiamenti delle esigenze di rete.
\end{itemize}
P4 è stato introdotto per superare le limitazioni di OpenFlow al fine di fornire una soluzione più flessibile.
\\OpenFlow, pur separando il piano di controllo e il piano dati, si basa su regole di elaborazione dei pacchetti attraverso tabelle che mappano i campi degli header (indirizzi IP, MAC, porte...) in un insieme fisso di funzionalità.
\\Negli anni le specifiche dei pacchetti sono diventate sempre più complesse, nonchè dipendenti dalle singole aziende che hanno iniziato a sviluppare l'hardware indipendentemente, rendendo necessari continui aggiornamenti per supportare le varie esigenze\cite{p4art}.
D'altra parte, gli switch non aggiornati, o prodotti da aziende diverse, non riescono a supportare tutte le nuove caratteristiche a causa delle limitazioni hardware.
\\Inoltre, OpenFLow non fornisce delle interfacce operative o amministrative standard, quindi rende complicato aggiungere supporto per nuovi protocolli.
Tutte queste problematiche hanno portato all'introduzione di dispositivi programmabili che utilizzano P4.
\\P4 consente di definire intestazioni e tabelle personalizzate, e programmare esplicitamente il flusso di controllo dello switch, permettendo di adattarsi rapidamente ai cambiamenti e alle innovazioni.
\begin{figure}[h]
    \centering
   \includegraphics[width=1\textwidth]{p4.png}
    \caption{Workflow di P4 sul piano dati \cite{p4Article}}
    \label{fig:p4}
\end{figure}
%Permette di definire la tabella, le azioni e the counters per poi essere applicate a elementi hardware o software switch allo stesso modo.
%P4 ha un interfaccia per analizzare pacchetti e match campi nell'header. In questo modo abbiamo un accoppiamento tra hardware e software.
\\Nella Figura \ref{fig:p4} è illustrato il Workflow del modello di P4.
\\Un programma P4 (P4 Program) definisce il comportamento del piano dati desiderato ed è suddiviso in varie sezioni, ognuna delle quali descrive un aspetto specifico del trattamento dei pacchetti.
Il processo inizia con la dichiarazione degli header del pacchetto da analizzare, successivamente si
definisce il comportamento del parser, le tabelle, le azioni e il flusso di controllo.
%Questi header contengono campi come indirizzi IP, numeri di porta .
\\Il programma è stato poi inviato al compilatore P4 (P4 Compiler) che genera due tipi di output. 
Il primo è un file eseguibile che descrive le varie operazioni da implementare all'interno del dispositivo target.
I target P4 sono dispositivi programmabili ottenibili tramite hardware, come ASICs o FPGA, o software, che realizzano il comportamento desiderato mettendo in atto le specifiche descritte nel programma grazie al file eseguibile.
%Permette inoltre di definire il piano dati in modo dinamico collegandolo al piano di controllo. 
Il parser all'interno dei dispositivi specifica come estrarre e interpretare i vari header dei pacchetti seguendo uno schema prestabilito.
\\La pipeline di elaborazione (Match-Action Pipeline) include tabelle e azioni che determinano come processare i pacchetti. %guarda se mettere separato tabelle e azioni
La struttura della pipeline è distinta per ciascun dispositivo ed è descritta da un determinato modello di architettura.
Il flusso di controllo invece coordona parser e pipeline per garantire il corretto funzionamento dell'intero processo.
Infine, il deparser ricompone i pacchetti con le eventuali modifiche degli header e successivamente li reintroduce nella rete.
\\Il secondo file generato dal compilatore è indipendente dal target e 
contiene le informazioni necessarie per far comunicare il piano di controllo e il piano dati tramite l'API P4Runtime.
\\P4Runtime permette al controller di connettersi ai dispositivi e interagire con la pipeline 
per poter inviare le configurazioni nella relativa tabella \cite{p4Article}. 
I dettagli hardware del piano dati sono nascosti al piano di controllo rendendolo indipendente dalle funzionalità e dai protocolli supportati.
%E' lo schema del pipeline descritto al controller in modo da nascondere il tipo di dispositivi presenti nella rete.
P4 rappresenta un passo significativo verso reti più flessibili e programmabili, consentendo agli sviluppatori di adattarsi rapidamente ai cambiamenti dei requisiti di rete.
Si propone come una soluzione innovativa e versatile per superare le limitazioni degli attuali protocolli e dispositivi di rete fornendo un linguaggio dinamico e indipendente dall'hardware.

\section{Descrizione Workflow}
Nei paragrafi successivi verrà descritto il Workflow della sperimentazione illustrato in Figura \ref{fig:componenti}. In seguito verranno 
forniti i dettagli più pratici relativi a un'analisi approfondita del codice e della documentazione delle funzioni del controller utilizzate relative ai servize e alle politiche.
\begin{figure}[h]
    \centering
   \includegraphics[width=1\textwidth]{componenti.png}
    \caption{Interazione tra alcune componenti in TeraFlow}
    \label{fig:componenti}
\end{figure}
\\\textbf{Servizio end-to-end}
\\Come già detto in precedenza il Device Level sfrutta una South-Bound Interface (SBI) per interagire con i device, in questo caso, tramite l'API P4Runtime. 
Inizialmente, il codice P4 compilato è stato copiato nel pod SBI per poter inserire le giuste configurazioni nelle tabelle dei dispositivi.
Il passo successivo consiste nel registrare i dispositivi e i link al controller SDN, in particolare alla componente di Context, per permettere una corretta comunicazione tra di essi.
A questo punto, siamo in grado di richiedere una connessione tra due end points alla componente di Service tramite un servizio.
\\Per mantenere il processo agnostico rispetto ai dettagli della tecnologia, la componente  
sfrutta una definizione minima, permettendo agli utenti di specificare solo quali sono i dispositivi finali che si vogliono connettere.
La componente traduce automaticamente la richiesta in modelli di configurazioni astratte dei dispositivi. 
Queste vengono a loro volta tradotte in regole P4 dal driver del dispositivo P4 della SBI.
\\La creazione del servizio è realizzata grazie alle funzioni messe a disposizione dalla componente di Service all'interno del controller la cui implementazione si può trovare nel file \cite{servserv}. 
\\Inizialmente, tramite la funzione \textit{CreateService}, si crea una servizio di connettività vuoto nel quale è stato specificato solo il tipo associato per poi memorizzare l'identificativo  
a cui è stato correlato nel Context database e ritornarlo.
Nella Figura \ref{fig:componenti} il servizio è stato richiesto dalla componente Web UI, mentre nella nostra sperimentazione si è usato uno script interno al controller.
Successivamente è stato aggiornato popolando i campi richiesti come endpoints, vincoli e configurazioni di servizio. 
Di questo si occupa la funzione \textit{UpdateService} alla quale va passato il servizio con i relativi parametri \cite{D32}.
La componente di Service, a questo punto, si rivolge prima alla componente di Context per recuperare la versione più aggiornata del servizio e settare lo stato a Planned (pianificato), poi alla PathComp per calcolare un percorso.
\\Per eseguire questa operazione la PathComp utilizza delle informazioni della rete che risiedono nel Context database.
%La PathComp è in grado di soddisfare anche richieste di servizio che attraversano più livelli, in questo caso, la risposta includerà uno o più sottoservizi con le relative sottoconnessioni che li supportano. \textbf{CAPISCILO SENNO LO TOGLI}
\\Una volta ricevute le connessioni la componente Service istanzia un \textit{Task Scheduler}. %che li correla e li schedula.
Quest'ultimo è responsabile dell'esecuzione delle attività di installazione e smantellamento dei servizi per garantire la gestione dei relativi collegamenti nell'ordine appropriato.
Come ultima operazione è stato eseguito il metodo \textit{Execute} del \textit{Task Scheduler} per realizzare tutte le operazioni di configurazione richieste per i dispositivi lungo il percorso tramite l'SBI; infine è stato modificato lo stato del servizio in Active (attivo).
\\Parallelamente, il database della componente Context è stato aggiornato con le nuove informazioni e l'identificatore del servizio è stato nuovamente restituito all'entità chiamante.
\\
\\\textbf{Vincoli e configurazioni di servizio}
\\Oltre alla specifica degli endpoints è possibile richiedere azioni supplementari come l'aggiunta di vincoli o configurazioni di servizio specifiche. 
%In questo modo il componente Policy offre SLA basati sugli eventi per servizi di connettività end-to-end tramite pipeline P4
Queste vengono specificate inizialmente nella fase di creazione di un servizio e devono essere rispettate finchè quest'ultimo non verrà eliminato. 
Ciò semplifica la gestione dei servizi, in quanto la dichiarazione di informazioni aggiuntive può sostituire l'associazione di una politica.
\\Nella demo presa in considerazione i vincoli non erano specificati ma sono stati aggiunti per quanto riguarda la latenza e la capacità del percorso, 
come si può vedere nella Figura \ref{fig:abilene}.
Questa aggiunta non ha prodotto cambiamenti nella manutenzione del servizio poiché le funzioni relative all'effettivo funzionamento di queste funzionalità non 
sono ancora state implementate nel controller ma sono solo pianificate per release future.
Anche se nella documentazione ufficiale non sono menzionati i vari tipi di vincoli supportati 
si possono ritrovare nel file \cite{vincoli}.
\\
\\\textbf{Politica}
\\Una parte fondamentale nei sistemi moderni è la gestione a run-time del servizio stabilito\cite{demo}.
Per garantire questa richiesta è necessario associare una politica al servizio per richiedere un Service Level Agreement (SLA).
\\Inizialmente, alla creazione del servizio, si associano le differenti Kpi tramite i KpiDescriptor.
In ogni KpiDescriptor  si devono specificare i valori numerici da rispettare insieme al tipo di Kpi (KpiSampleType), che può essere predefinito (illustrati nel File \cite{kpi}),
oppure, utilizzando il tipo UNKNOWN, si può personalizzare tramite una descrizione.
Successivamente si richiama la funzione della componente di Monitoring \textit{SetKpi} per ogni regola, passando il KpiDescriptor come parametro, così da
associare il monitoraggio delle metriche richieste a una kpi nel Monitoring database e restituire il relativo identificatore.
Se a uno stesso servizio si associano più KPI ognuna deve avere associato un KpiSampleType diverso, altrimenti, anche se con una descrizione differente,
verrà associato lo stesso identificativo per monitorare metriche diverse creando conflitti. 
Infine al servizio è stata associata una politica specificando le diverse regole collegate tra loro da operatori booleani come AND/OR tramite file JSON.
Per definire una politica va specificato l'id del contesto a cui si vuole associare e l'id del relativo servizio, successivamente si specificano le regole.
Ogni regola è composta dall'identificatore della Kpi, l'operatore numerico (maggiore, minore o uguale \cite{op}) insieme al valore limite 
per definire l'intervallo di valori non ammessi e infine l'azione da eseguire (le possibili azioni sono riportare nel file \cite{az}).
Quando i requisiti non vengono soddisfatti, la componente di Monitoring solleva un allarme che fa scattare l'azione prestabilita dal servizio.
%\\Nella sperimentazione abbiamo introdotto una politica per il servizio stabilito tramite un file JSON
%con tre condizioni: una per la latenza (che non deve superare i 100ms), una per la loss (che non deve superare il 5\%) 
%e infine una per la capacità\ref{fig:policy}.
%Queste tre condizioni sono legate tra loro tramite un "OR", quindi appena una di esse non è più rispettata viene invocata l'azione che in questo caso consiste nel ricalcolo del percorso.
%MODIFICA O LA FOTO O LA DESCRIZIONE

\section{Sperimentazione}
In questa sezione verranno descritti gli esperimenti svolti esponendo i codici, le topologie e i comandi usati 
in modo tale da permetterne la riproduzione. In alcuni casi è stata necessaria la modifica dei codici originali; tutti i file sono riportati in Appendice.
\\Per verificare il comportamento del controller, non avendo a disposizione una rete reale, sono state utilizzate delle topologie
di rete riprodotte tramite Mininet basate su bmv2\cite{bmv2}, la seconda versione dello switch software di riferimento P4.
\subsection{Demo}
Inizialmente si è riprodotta la demo dell'Hackfest 3;
i comandi e la spiegazione completa si possono ritrovare nella pagina del sito \cite{hackfest}.
\\Dopo aver verificato che le componenti del controller siano in stato di running (\textit{Kubectl get pods -n=tfs}),
si istanzia la topologia \cite{topo4}, illustrata in Figura \ref{fig:top4}, sul container di Mininet collegato al controller.
\begin{figure}[h]
    \centering
    \includegraphics[width=1\textwidth]{topologiasw4.png}
    \caption{Topologia 4 switch e 2 percorsi}
    \label{fig:top4}
\end{figure}
Dalla cartella \textit{ngsdn-tutorial} si invoca \textit{make start} e \textit{make mn-cli} per l'avvio.
\\Nel file Objects.py sono descritti gli switch che fanno parte della topologia con i relativi
dispositivi, collegamenti e servizi che verranno usati dai vari script per reperire le informazioni necessarie.
\\Il primo comando da eseguire sul terminale relativo al controller (\textit{./src/tests/hackfest3/setup.sh}) configura il pod Kubernetes dell'SBI
inserendo le configurazioni di P4 necessarie al server per comunicare con gli switch.
\\Successivamente tramite il comando \textit{./src/tests/hackfest3/run\_test\_01\_bootstrap.sh} è stata registrata la topologia collegata a Mininet al controller,
inserendo i dispositivi e i collegamenti nel Context database, così da poter proseguire con l'installazione del servizio.
\\Cercando nel browser http://localhost:8080/webui si può accedere all'interfaccia grafica e, selezionando il contesto \textit{Context:(admin):Topology(admin)},
si può visualizzare la topologia di rete e le configurazioni dei device.
\\Prima di continuare con la creazione del servizio, in un altro terminale, dalla cartella
\textit{probe-tfs}, si copia ricorsivamente la cartella target nella directory corrente (\textit{cp -r \textasciitilde /controller /src/tests/p4/probe/probe-tfs/target . })
necessaria per eseguire i comandi successivi.
Dallo stesso terminale ci si collega al container di Mininet (\textit{./connect-to-mininet.sh}) e
si attiva lo script \textit{tfsagent} (\textit{./tfsagent.sh}) che permette di ascoltare gli eventi dalla componente di Context e, 
quando un servizio viene registrato, crea la KpiDescriptor relativa alla latenza e la registra nel Monitoring database.
In questo script il KpiSampleType specificato è UNKNOWN e il relativo commento è "\textit{Latency value for service \{\}}".
Successivamente si mette in attesa dei dati dal tfsping per ricevere le metriche relative alla latenza del collegamento e crea delle kpi da mandare alla componente di Monitoring.
\\A questo punto si può creare il servizio tra i due endpoints (\textit{./src/tests/hackfest3/run\_test\_02\_ create\_service.sh}) e visualizzarlo sull'interfaccia web, come si può vedere nelle Figure \ref{fig:ser} \ref{fig:sw4}.
\begin{figure}[h]
    \centering
    \includegraphics[width=1\textwidth]{service active.png}
    \caption{Servizio}
    \label{fig:ser}
\end{figure}
\begin{figure}[h]
    \centering
    \includegraphics[width=1\textwidth]{policy sw2.png}
    \caption{Servizio 4 sw iniziale}
    \label{fig:sw4}
\end{figure}
Gli endpoints relativi a questa demo sono “SW1-port3” e “SW4-port3”, 
gli switch della topologia collegati rispettivamente al client e al server instanziati su Mininet.
A questo punto il comando \textit{client ping server} eseguito sul terminale iniziale di Mininet funziona.
Per continuare con la parte di politica si modifica il file \textit{manifests/policyservice.yaml}, come rappresentato in Figura \ref{fig:pol}, aprendolo dal terminale con il comando \textit{kubectl edit svc policyservice -n=tfs}.
\begin{figure}[h]
    \centering
    \includegraphics[width=0.7\textwidth]{modificafile.png}
    \caption{File policyservice modificato}
    \label{fig:pol}
\end{figure}
Adesso, da un ulteriore terminale, dalla cartella \textit{grpc}, si può installare la regola di politica tramite un file JSON (\textit{policyAddService.json}) eseguendo \textit{./addPolicy.sh}.
Dalla Web UI si può visualizzare la politica rappresentata nella Figura \ref{fig:lat}.
\begin{figure}[h]
    \centering
    \includegraphics[width=1\textwidth]{politica lat.png}
    \caption{Politica con regola di latenza}
    \label{fig:lat}
\end{figure}
Infine, per verificare che la politica inserita sia efficace, tramite il primo terminale aperto su Mininet si 
aggiunge un ritardo su un'interfaccia di uno switch interno al percorso prestabilito tramite il comando
\textit{switch2 tc qdisc add dev switch2-eth2 root netem delay 2000ms}, in questo caso è lo switch2 sull'interfaccia eth2.
A questo punto si esegue \textit{client ./tfsping}, uno script che, richiamando il comando ping, calcola la latenza del percorso per mandarla al \textit{tfsagent} e si attende un cambiamento del percorso.
In questa fase, finchè il percorso non è cambiato e la latenza continua a essere superiore a quella richiesta,
lo stato della politica passa a ACTIVE a PROVISIONED o UPDATED, mentre il servizio a ENFORCED.
Nella Figura \ref{fig:sw4 dopo} si può vedere il nuovo percorso. 
\\Si è fatta anche un'ulteriore prova, invece di aumentare la latenza si è direttamente disattivato un link tra due switch, in questo caso switch2 e switch4,
tramite \textit{link switch2 switch4 down} e si è potuto constatare che anche in questo caso è stata attivata l'azione della politica.
\begin{figure}[h]
    \centering
    \includegraphics[width=1\textwidth]{policy con sw3.png}
    \caption{Servizio 4 sw cambio percorso}
    \label{fig:sw4 dopo}
\end{figure}
Prima di spengere la macchina virtuale del controller ci si deve assicurare di disattivare correttamente le risorse istanziate.
Per prima cosa va rimossa la politica dalla quale dipende il servizio \textit{./removePolicy.sh}. Dopo aver eseguito il comando 
è opportuno verificare tramite l'interfaccia web che la politica non è più disponibile; vista l'instabilità del controller potrebbe essere necessario rieseguirlo una seconda volta prima di continuare.
Successivamente si disattiva il servizio (\textit{./src/tests/hackfest3/run\_test\_03\_ delete\_service.sh}) e si rimuovono i dispositivi e l'ambiente allocati (\textit{./src/tests/hackfest3/run\_test\_04\_ cleanup.sh}).
%Inoltre, si è notato che una volta rimosso il ritardo, la perdita di pacchetti, o riattivato un link, il controller percepisce questo cambiamento e modifica automaticamente il percorso precedente, poiché è memorizzato nel database della Path Comp come percorso preferito per le sue caratteristiche.
\subsection{Esperimento 2}
In questo esperimento la topologia di base da cui ha avuto inizio è stata quella descritta nel file \textit{controller/src/tests/p4/mininet/8switch3path.py}.
Dopo sono stati aggiunti dei link per creare 5 possibili percorsi al posto di 3; la topologia finale è illustrata nella Figura \ref{fig:top8}. 
\begin{figure}[h]
    \centering
    \includegraphics[width=1\textwidth]{sw4.drawio.png}
    \caption{Topologia 8 switch e 5 percorsi}
    \label{fig:top8}
\end{figure}
\\Prima di inizializzare la topologia è stato necessario modificare il file \textit{ngsdn-tutorial/mininet/docker-compose.yaml}
per abilitare ulteriori porte (50005,50006,50007,50008) su Mininet per il collegamento al controller.
Per quanto riguarda il terminale Mininet l'unico comando che cambia è \textit{make start-8} per inizializzare questa nuova topologia.
\\Gli script necessari si possono trovare nella cartella P4 del controller \cite{ofc} dove è stata necessaria l'aggiunta dei nuovi collegamenti al file \textit{Objects.py} per una corretta sincronizzazione con il container Mininet. 
Nello stesso file sono stati aggiunti dei vincoli di configurazione relativi alla latenza e alla capacità, con la conseguente modifica anche del file \textit{test\_functional\_create\_service.py}.
Infine, il file \textit{./src/tests/p4/setup.sh} è stato standardizzato con il relativo nella cartella hackfest3 per un'errore nella progettazione. La modifica 
consiste nell'aggiunta di una riga di codice che consente di eliminare il pod relativo al server P4. Questa modifica non è necessaria se l'esperimento è svolto solamente una volta.
\\Per quanto riguarda la parte iniziale i comandi sono gli stessi, ovviamente cambiando la cartella dei file, quindi \textit{./src/tests/p4/setup.sh} e \textit{./src/tests/p4/run\_test\_01\_bootstrap.sh}.
\\In seguito, per il probe composto da agent e ping, si è continuata la demo dell'Hackfest 3 con la parte dell'Interactive Hacking Session \#1.
In questa parte il probe si trova nella cartella \textit{controller/src/tests/hackfest3/new-probe/solution} e sono due script scritti in Python con il medesimo funzionamento dei precedenti.
\\I file sono stati poi modificati per aggiungere la Kpi relativa alla packet loss. Per il funzionamento dell'agent
è stato necessario l'aggiunta del KpiSampleType per la latenza. Pur essendo uno dei tipi predefiniti, non tutti sono implementati completamente, infatti
i KpiSampleType in stato finale sono: UNKNOWN, PACKETS\_TRANSMITTED, PACKETS\_RECEIVED, BYTES\_TRANSMITTED, BYTES\_RECEIVED e LINK\_TOTAL\_CAPACITY \_GBPS.
Per questo motivo all'inizio la Kpi relativa alla latenza non aveva l'effetto desiderato; per sistemare questo problema è stata necessaria la 
modifica di alcuni file di configurazione del controller relativi alla gestione dei kpiSampleType.
La stessa cosa non è stata possibile per la packet loss in quanto non era tra i tipi, anche solo parzialmente, implementati. 
\\In un terminale relativo alla cartella si eseguono i seguenti comandi: \textit{source \~/tfs-ctrl/tfs\_ runtime\_env\_vars.sh} per inserire le variabili di TeraFlow
necessarie, se l'ambiente Python è disattivato \textit{pyenv activate 3.9.18/envs/tfs}, e infine \textit{python agent.py}.
Quando si eseguirà il comando relativo alla creazione del servizio (\textit{./src/tests/p4/run\_test\_02\_create\_service.sh}),
sul terminale in cui è in esecuzione l'agent si vedrà l'output in Figura \ref{fig:ag}.
\\Il servizio richiesto ha come endpoints "SW1-port4” e “SW8-port4”, come si può vedere in Figura \ref{fig:sw8}.
\begin{figure}[h]
    \centering
   \includegraphics[width=1\textwidth]{service sw8.png}
    \caption{Servizio 8 sw iniziale}
    \label{fig:sw8}
\end{figure}
\begin{figure}[h]
    \centering
   \includegraphics[width=0.5\textwidth]{create new event ag.png}
    \caption{Creazione kpi}
    \label{fig:ag}
\end{figure}
\\Per creare la politica si riutilizzano gli script precedenti assicurandosi di cambiare nel file \textit{policyAddService.json}
gli id relativi al contesto e al servizio e di aggiungere alle regole quella relativa alla packet loss.
La politica risultante è illustrata in Figura \ref{fig:policy}.
\begin{figure}[h]
    \centering
   \includegraphics[width=1\textwidth]{policy 5 100.png}
    \caption{Politica latenza e pkt loss}
    \label{fig:policy}
\end{figure}
\\Sul terminale Mininet, oltre alle prove fatte aggiungendo dei ritardi e disattivando dei link si è provato a simulare una packet loss tramite il comando
\textit{switch4 tc qdisc add dev switch6-eth2 root netem loss 7\%}.
Per verificare il cambio del percorso si è dovuto modificare anche il file \textit{ping2.py} per riuscire a mandare tramite la socket
entrambe le metriche, calcolate grazie all'esecuzione del comando ping di Mininet. 
Per un problema di comunicazione si è modificato il percorso della socket, infatti inizialmente era \textit{/home/teraflow/ngsdn-tutorial/tmp/sock} ed
è stato cambiato in \textit{"/home/teraflow/ngsdn-tutorial/tmp/tfsping"}.
A questo punto si è attivato lo script con
i comandi eseguiti dalla cartella \textit{solution}: \textit{./copy.sh};
\textit{cd \~/ngsdn-tutorial}; \textit{make mn-cli}; \textit{client python ping2.py 10.0.0.2}.
Anche in questo caso si è riscontato un cambiamento nel persorso come mostrato in Figura \ref{fig:sw8 dopo}.
Infine tramite i comandi esposti in precendenza, cambiando nello script \textit{removePolicy.json} l'identificativo della politica, si sono disattivati in ordine la politica, il servizio e i dispositivi.
\begin{figure}[h]
    \centering
   \includegraphics[width=1\textwidth]{servizion constraints.png}
    \caption{Servizio 8 sw cambio percorso}
    \label{fig:sw8 dopo}
\end{figure}

\subsection{Esperimento 3}
L'ultima topologia utilizzata è stata Abilene \cite{abilene}; una rete di trasporto creata da Internet2.
La topologia Mininet si può trovare su \cite{topab}; per la simulazione si sono cambiati i valori di banda a 30
e si sono effettuati dei cambiamenti per la connessione al container e relativi al servizio, come il settaggio ARP per i due endpoints. 
Per poter utilizzare la topologia sono state aggiunte ulteriori porte a Mininet (5009, 5010, 5011) ed è stato modificato il 
make file della cartella \textit{ngsdn-tutorial} per poter permettere l'inizializzazione tramite il comando \textit{make start-ab}.
\\Per quanto riguarda gli script sono stati riutilizzati quelli della sperimentazione precedente nella partella P4 \cite{ofc} con le 
modifiche necessarie al file \textit{Objects.py}. 
I comandi iniziali sono i medesimi.
\\In questa sperimentazione è stata introdotta una nuova regola di politica relativa alla capacità.
A tal fine è stato modificato nuovamente il file \textit{agent.py} con l'aggiunta di una nuova KpiDescriptor, con il relativo KpiSampleType LINK\_TOTAL\_CAPACITY\_GBPS, e si è avviato.
Si è creato il servizio, illustrato in Figura \ref{fig:abilene}, con endpoints "SW1-port3” e “SW8-port4” che nella topologia corrispondono alle città di New York e Kansascity.
Successivamente la politica, raffigurata in Figura \ref{fig:policy cap}, cambiando nel file JSON i campi relativi
al contesto, al servizio e alle regole.
\begin{figure}[h]
    \centering
   \includegraphics[width=1\textwidth]{service abilene (10).png}
    \caption{abilene iniziale}
    \label{fig:abilene}
\end{figure}
\begin{figure}[h]
    \centering
   \includegraphics[width=1\textwidth]{policy cap.png}
    \caption{Servizio 8 sw cambio percorso}
    \label{fig:policy cap}
\end{figure}
Infine per verificarla , oltre ai test effettuati anche negli altri esperimenti, 
si è modificata la capacità tramite due differenti comandi: \textit{switch4 tc qdisc add dev switch4-eth2 root tbf rate 1mbit burst 10kb latency 50ms} che 
limita la velocità di trasmissione a 1mbit per secondo e imposta un ritardo massimo di 50ms per i pacchetti nella coda; 
\textit{switch1 tc qdisc add dev switch1-eth1 root tbf rate 1mbit burst 10kb limit 10000} che
limita la velocità di trasmissione e specifica un limite sulla dimensione massima della coda a 10.000 byte.
Prima di attivare lo script del ping vanno effettuate delle modifiche.
Mentre la latenza e la packet loss potevano essere calcolate tramite il comando ping, per la capacità questo non è possibile.
Per calcolare la nuova metrica si è introdotto nel codice la chiamata anche al comando iperf.
Ora, prima dell'esecuzione del codice su Mininet (\textit{h0 python ping2.py 10.0.0.8}), va attivato iperf sul dispositivo finale \textit{h7 iperf -s \&} (in questo caso si è scelto di attivarlo in background per riutilizzare lo stesso terminale).
Anche in quest'ultima sperimentazione il percoso è cambiato come si può vedere nelle Figure \ref{fig:abilene dopo} \ref{fig:abilene post}.
Come negli altri esperimenti, con le solite modifiche ai file, si sono disattivati politica, servizio e dispositivi per ricreare l'ambiente iniziale.
\begin{figure}[h]
    \centering
   \includegraphics[width=1\textwidth]{service abilene (11).png}
    \caption{Servizio 8 sw cambio percorso}
    \label{fig:abilene dopo}
\end{figure}
\begin{figure}[h]
    \centering
   \includegraphics[width=1\textwidth]{service 3 10 11.png}
    \caption{abilene cambio }
    \label{fig:abilene post}
\end{figure}


