\chapter{Controller SDN}
\label{ch:SDN}

L'architettura tradizionale di rete si basa su dispositivi fisici di interconnessione, switch e router, che collegano più host a livello locale e permettono di scambiarsi informazioni.
Questi dispositivi, situati al livello rete dello stack TCP/IP, implementano al loro interno sia funzioni del piano di controllo (control plane),
ossia l'instradamento dei datagrammi, che funzioni del piano dati (data plane) che si occupa dell' inoltro del traffico. Tali operazioni sono basate sulla destinazione
e sono effettuate grazie alle decisioni della parte controllo che ha il compito di aggiornare la tabella di inoltro. 
In questo caso si hanno due tipi di algoritmi di instradamento: 
\begin{itemize}
\item decentralizzati: nessun nodo conosce la topologia di tutta la rete ma ha informazioni solo dai nodi vicini. 
\item globali: si basano sulla conoscenza della topologia di tutta la rete. Questa soluzione
porta allo scambio di molti messaggi di broadcast e non è efficiente in quanto ogni
nodo si occupa di calcolare soltanto la propria tabella.
\end{itemize}
Entrambi gli approcci necessitano di un algoritmo di routing in esecuzione su ciascun router che, attraverso un apposito protocollo, 
scambia messaggi con gli altri componenti della rete per prendere decisioni. Questo introduce ritardi non rendendo la rete adatta alle nuove esigenze delle applicazioni moderne altamente dinamiche.
%Questo tipo di reti utilizzano controller centralizzati
%il livello rete interconnette reti eterogenee
%Una limitazione dei controller centralizzati infatti è la configurazione statica tra gli switch e il controller.
\\Il Software Defined Networking (SDN) è un'architettura di rete proposta negli ultimi anni dalla Open Networking Foundation (ONF) \cite{ONF} per rimediare ai problemi di scalabilità e affidabilità dei controller decentralizzati.
SDN è un nuovo paradigma dinamico, gestibile e facilmente adattabile grazie alla separazione del piano di controllo dal piano dati rendendo così possibile un disaccoppiamento tra hardware e software per la gestione di device con API diverse.
\\La base di questo paradigma è un controller remoto che, interagendo con i Control Agents
locali, riceve informazioni sui collegamenti e sul traffico in tempo reale ed è in grado di
configurare autonomamente i device collegati sulla base degli eventi notificati. Lo scopo
principale è quindi ridurre e semplificare il carico di amministrazione.
\begin{figure}[h]
    \centering
   \includegraphics[width=1\textwidth]{sdn.png}
    \caption{struttura di una rete SDN}
    \label{fig:sdn}
\end{figure}
%foto  https://community.fs.com/it/article/what-is-software-defined-networking-sdn.html
Come si può notare dalla figura \ref{fig:sdn} la rete viene suddivisa in tre livelli: Infrastructure layer, Control layer e Application layer.
Partendo dal livello più basso troviamo l'infrastruttura di rete il cui unico compito è implementare il piano dati. 
Questa divisione consente di evitare algoritmi
di routing per il forwarding dei pacchetti all'interno dei dispositivi di rete visto che sarà
gestito dai livelli sovrastanti. I dispositivi dovranno solo essere in grado di comunicare gli
eventi che si verificano nella rete al controller.
\\Nel control layer viene situato il controller SDN che tramite API northbound (NBI) e southbound (SBI), tipicamente implementata tramite OpenFlow, permette di comunicare con
gli altri due livelli. Questo livello consente il monitoraggio e l'implementazione del piano
dati. Il controller è logicamente centralizzato, anche se fisicamente può essere distribuito, permettendo al piano di gestione (management plane) sovrastante di interagire con un
unico punto di controllo. E' in grado di configurare e di gestire rapidamente le risorse di rete tramite programmi dinamici e automatizzati. Consente infatti di imporre regole di
inoltro ai dispositivi sottostanti, dopo aver calcolato il percorso migliore, tramite la manipolazione delle tabelle di routing mediante delle API SBI.
\\L'application layer comprende le applicazioni e i servizi che sfruttano le capacità della
rete SDN per la realizzazione del piano di gestione. Grazie a questo livello si possono
definire policy da implementare all'interno della rete. Queste regole sono comunicate al
controller tramite le API NBI e quest'ultimo si occuperà di farle rispettare mediante il costante monitoraggio delle risorse. 
Questo disaccoppiamento dei vari livelli, al contrario dell'approccio
tradizionale, permette alla rete di diventare direttamente programmabile da un'unica unità
centrale che mantiene una visione globale e di astrarre l'infrastruttura sottostante ai vari servizi per poter far fronte alle difficoltà di gestione che si sono ritrovate nelle reti
moderne.
%\section{OpenFlow}
%https://opennetworking.org/wp-content/uploads/2013/05/TR-535_ONF_SDN_Evolution.pdf  pagine 8
