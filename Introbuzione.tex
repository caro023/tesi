\chapter{Introduzione}
L'evoluzione delle reti moderne e l'incremento delle esigenze di prestazioni e affidabilità, insieme con l'aumento della quantità di traffico, 
hanno portato alla necessità di nuovi paradigmi di gestione e controllo delle reti.
%\textit{Servizi altamente esigenti come il 5G/6G si sono insinuati in questo contesto.}
\\In questo contesto, il Software Defined Networking (SDN)\ref{ch:SDN} rivoluziona la gestione delle reti attraverso un approccio centralizzato e programmabile, separando il piano di controllo dal piano dati.
Questa separazione permette agli amministratori di rete di semplificare configurazioni complesse e di introdurre politiche di rete dinamiche e affidabili.
\\Tuttavia, le funzioni degli ambienti SDN sono ancora focalizzate principalmente sul controllo del funzionamento dei dispositivi. Ad esempio, si possono impostare regole come \textit{"consenti il traffico tra X e Y solo se appartiene alla VLAN Z"} per gestire il flusso dei dati. 
Sebbene questo approccio semplifichi la gestione dei dispositivi, non affronta completamente la complessità della gestione delle reti su larga scala.
%Tuttavia le funzioni degli ambienti SDN riguardano il funzionamento dei dispositivi, come ad esempio \textit{"consenti il traffico tra X e Y solo se appartiene alla VLAN Z"}, andando a impostare
%le regole per consentire il flusso desiderato.
\\D'altra parte, l’Intent-Based Networking (IBN) \ref{ch:IBN}, o Networking basato su intenti, è un paradigma che mira a separare ulteriormente la complessità di implementazione dal livello di gestione. 
%Questa metodologia permette di affrontare le nuove sfide per la gestione delle reti.
%L'idea fondamentale dell'IBN è di poter esprimere i propri obiettivi per il comportamento della rete specificando intenti ad alto livello.
IBN affronta le sfide emergenti fornendo un'astrazione di alto livello dove gli obiettivi per il comportamento della rete possono essere espressi in modo chiaro e intuitivo.
Ad esempio, un intento potrebbe essere: \textit{"consentire alle applicazioni contabili di accedere al server XYZ, ma non consentire l’accesso alle applicazioni di produzione"} oppure
\textit{"mantenere sempre un'elevata qualità del servizio e un'elevata larghezza di banda per gli utenti di un certo livello"}.
 Il framework IBN si occupa di tradurre questi intenti in configurazioni di rete specifiche e di adattare dinamicamente la rete per rispettarli, gestendo automaticamente le complessità sottostanti.
%Il framework IBN si occuperà di elaborare questi intenti e apportare le modifiche necessarie nella configurazione dei dispositivi di rete affichè essi vengano rispettati. 
Questo approccio consente alla rete di adattarsi dinamicamente sulla base dei cambiamenti in tempo reale, gestendo le difficoltà in modo autonomo.
%Questa astrazione di secondo livello è la differenza principale quando si tratta di reti basate su intenti rispetto a SDN.
\\L'obiettivo della tesi è l'implementazione di un intento in Teraflow, uno dei più recenti controller SDN.
Inizialmente è stato necessario comprendere l'architettura del controller, incluse le sue componenti, per capire le loro interazioni con l'ambiente circostante e la gestione dei servizi e delle politiche.
Per capire meglio il funzionamento del software, sono stati eseguiti dei casi di test forniti dagli amministratori per simulare e valutare il comportamento del sistema in contesti applicativi specifici.
Successivamente, si è passati alla fase di sperimentazione.
Non avendo a disposizione una rete reale, è stato utilizzato l'emulatore di reti Mininet \ref{ch:Mininet} come ambiente di test.
\\Partendo da file esistenti, si è creato il contesto necessario e un servizio tra due end-points.
Al servizio è stata associata una politica per mantenere i livelli di loss ratio e latenza sotto una determinata soglia.
Alla politica viene associata un'azione nel caso le richieste non fossero più rispettate, in questo caso è stato scelto il ricalcolo del percorso come azione.
Per monitorare dinamicamente in tempo reale la qualità del servizio, è stato implementato un probe che, attraverso il ping tra i due end-point, misura le due metriche e invia 
i relativi valori.
\\Il resto del documento sarà organizzato come segue.
\\Nel capitolo \ref{cap:contesto} verrà esposto il contesto. All'interno ci sarà un'analisi approfondita del Software Defined Networking (SDN)
e verranno illustrati alcuni dei principali controller SDN presenti sul mercato. Parallelamente nello stesso capitolo verrà, introdotto anche l'Intent-Based Networking (IBN).
\\Nel capitolo \ref{cap:teraflow} verrà descritto in modo approfondito il controller Teraflow su cui si incentra la tesi.
Verrà delineata l'architettura e l'interazione necessaria tra le componenti affinchè la politica venga rispettata.
Saranno inoltre introdotti gli strumenti utilizzati nella sperimentazione, tra cui P4, gRPC e Mininet, spiegando il loro scopo.
Nel capitolo \ref{cap:policy} viene descritta, dopo aver introdotto la teoria necessaria, la parte pratica e il fulcro di questo lavoro.
Infine, nell’ultimo capitolo, verranno presentate le conclusioni.
