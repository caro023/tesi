\chapter{Appendice}

\section {Installazione}
Il controller viene eseguito all'interno di una Java Virtual Machine (JVM), quindi è necessario verificare quali versioni di Java supporta la distribuzione che si decide di installare.
La versione stabile più recente di ODL è compatibile con le versioni di Java superiori alla 17, mentre per le versioni più datate quest'ultime non vanno bene. 
Per l'installazione si deve scaricare la distribuzione desiderata che si trova sul loro sito ufficiale nella pagina di download del software \cite{InstallODL}.
Successivamente è necessario fare l'unzip del file, navigare nella cartella e eseguire il seguente comando da terminale per avviare il controller.
\begin{lstlisting}[language=CLI]
./bin/karaf
\end{lstlisting}
La versione di ODL dell'immagine \ref{fig:installazione} è Potassium.
\begin{figure}[h]
    \centering
   \includegraphics[width=1\textwidth]{odl/Potassium.png}
    \caption{Installazione ODL}
    \label{fig:installazione}
\end{figure}
\\Successivamente, si può trovare una lista completa delle feature disponibili eseguendo il seguente comando.
\begin{lstlisting}[language=CLI]
feature:list
\end{lstlisting}
\begin{figure}[h]
    \centering
   \includegraphics[width=1\textwidth]{odl/feature.png}
    \caption{Alcune features disponibili}
    \label{fig:feature}
\end{figure}
Per installarle invece 
\begin{lstlisting}[language=CLI]
feature:install <feature1> <feature2>..
\end{lstlisting}
