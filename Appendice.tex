\chapter{Appendice}

\section {Installazione ODL}
Il controller viene eseguito all'interno di una Java Virtual Machine (JVM), quindi è necessario verificare quali versioni di Java supporta la distribuzione che si decide di installare.
La versione stabile più recente di ODL è compatibile con le versioni di Java superiori alla 17, mentre per le versioni più datate quest'ultime non vanno bene. 
Per l'installazione si deve scaricare la distribuzione desiderata che si trova sul loro sito ufficiale nella pagina di download del software \cite{InstallODL}.
Successivamente è necessario fare l'unzip del file, navigare nella cartella e eseguire il seguente comando da terminale per avviare il controller.
\begin{lstlisting}[language=CLI]
./bin/karaf
\end{lstlisting}
La versione di ODL dell'immagine \ref{fig:installazione} è Potassium.
\begin{figure}[h]
    \centering
   \includegraphics[width=1\textwidth]{odl/Potassium.png}
    \caption{Installazione ODL}
    \label{fig:installazione}
\end{figure}
\\Successivamente, si può trovare una lista completa delle feature disponibili eseguendo il seguente comando.
\begin{lstlisting}[language=CLI]
feature:list
\end{lstlisting}
\begin{figure}[h]
    \centering
   \includegraphics[width=1\textwidth]{odl/feature.png}
    \caption{Alcune features disponibili}
    \label{fig:feature}
\end{figure}
Per installarle invece 
\begin{lstlisting}[language=CLI]
feature:install <feature1> <feature2>..
\end{lstlisting}

\section{Topologia mininet 8sw aggiunta link}
\label{cap:link}
\subsection{File Mininet}
\begin{lstlisting}[language=Python]
        self.addLink(switch2, switch5) 
        self.addLink(switch4, switch3)  
        
        # Host links
        self.addLink(client, switch1)   # Switch1: port 4
        self.addLink(server, switch8)   # Switch8: port 4
\end{lstlisting}
\subsection{File Objects.py}
\begin{lstlisting}[language=Python]
# SW2_3 - SW5_3
LINK_SW2_SW5_UUID           = get_link_uuid(ENDPOINT_ID_SW2_3, ENDPOINT_ID_SW5_3)
LINK_SW2_SW5_ID             = json_link_id(LINK_SW2_SW5_UUID)
LINK_SW2_SW5                = json_link(LINK_SW2_SW5_UUID, [ENDPOINT_ID_SW2_3, ENDPOINT_ID_SW5_3])

# SW5_3 - SW2_3
LINK_SW5_SW2_UUID           = get_link_uuid(ENDPOINT_ID_SW5_3, ENDPOINT_ID_SW2_3)
LINK_SW5_SW2_ID             = json_link_id(LINK_SW5_SW2_UUID)
LINK_SW5_SW2                = json_link(LINK_SW5_SW2_UUID, [ENDPOINT_ID_SW5_3, ENDPOINT_ID_SW2_3])

# SW4_3 - SW3_3
LINK_SW4_SW3_UUID           = get_link_uuid(ENDPOINT_ID_SW4_3, ENDPOINT_ID_SW3_3)
LINK_SW4_SW3_ID             = json_link_id(LINK_SW4_SW3_UUID)
LINK_SW4_SW3                = json_link(LINK_SW4_SW3_UUID, [ENDPOINT_ID_SW4_3, ENDPOINT_ID_SW3_3])

# SW3_3 - SW4_3
LINK_SW3_SW4_UUID           = get_link_uuid(ENDPOINT_ID_SW3_3, ENDPOINT_ID_SW4_3)
LINK_SW3_SW4_ID             = json_link_id(LINK_SW3_SW4_UUID)
LINK_SW3_SW4 
\end{lstlisting}

\section{Aggiunta costraints}
\subsection{File Objects}
\label{cap:obj}
\begin{lstlisting}[language=Python]
SERVICE_SW1_SW8_UUID          = get_service_uuid(ENDPOINT_ID_SW1_4, ENDPOINT_ID_SW8_4)
SERVICE_SW1_SW8               = json_service_p4_planned(SERVICE_SW1_SW8_UUID)
SERVICE_SW1_SW8_ENDPOINT_IDS  = [DEVICE_SW1_ENDPOINT_IDS[3], DEVICE_SW8_ENDPOINT_IDS[3]]
constraints     = [
                {"sla_capacity": {"capacity_gbps": 10.0}},
                {"sla_latency": {"e2e_latency_ms": 15.2}}
            ]
SERVICES = [
    (SERVICE_SW1_SW8, SERVICE_SW1_SW8_ENDPOINT_IDS, constraints),
]
\end{lstlisting}
\subsection{File test\_functional\_create\_service.py}
\begin{lstlisting}[language=Python]
    for service, endpoints, constraints in SERVICES:
        # Insert Service (table entries)
        service_uuid = service['service_id']['service_uuid']['uuid']
        print('Creating Service {:s}'.format(service_uuid))
        service_p4 = copy.deepcopy(service)
        service_client.CreateService(Service(**service_p4))
        service_p4['service_endpoint_ids'].extend(endpoints)
        service_p4['service_constraints'].extend(constraints)
        service_client.UpdateService(Service(**service_p4))
\end{lstlisting}
\section{Modifica setup in p4}
\begin{lstlisting}[language=Python]
export POD_NAME=$(kubectl get pods -n=tfs | grep device | awk '{print $1}')

kubectl exec ${POD_NAME} -n=tfs -c=server -- rm -rf /root/p4
kubectl exec ${POD_NAME} -n=tfs -c=server -- mkdir /root/p4

kubectl cp src/tests/p4/p4/p4info.txt tfs/${POD_NAME}:/root/p4 -c=server
kubectl cp src/tests/p4/p4/bmv2.json tfs/${POD_NAME}:/root/p4 -c=server
\end{lstlisting}

\section{Script policyAddService packet loss}
\label{cap:pktl}
\begin{lstlisting}[language=json]
{
    "serviceId": {
        "context_id": {
            "context_uuid": {
                "uuid": "43813baf-195e-5da6-af20-b3d0922e71a7"
            }
        },
        "service_uuid": {
            "uuid": "be1fbc16-fad4-5832-9a32-754a7d134b81"
        }
    },
    "policyRuleBasic": {
        "priority": 0,
        "policyRuleId": {
            "uuid": {
                "uuid": "1"
            }
        },
        "booleanOperator": "POLICYRULE_CONDITION_BOOLEAN_OR",
        "policyRuleState": {
            "policyRuleStateMessage": ""
        },
        "actionList": [
            {
                "action": "POLICY_RULE_ACTION_RECALCULATE_PATH",
                "action_config": [
                    {
                        "action_key": "",
                        "action_value": ""
                    }
                ]
            }
        ],
        "conditionList": [
            {
                "numericalOperator": "POLICYRULE_CONDITION_NUMERICAL_GREATER_THAN",
                "kpiValue": {
                    "floatVal": 5
                },
                "kpiId": {
                    "kpi_id": {
                        "uuid": "1"
                    }
                }
            },
            {
                "numericalOperator": "POLICYRULE_CONDITION_NUMERICAL_GREATER_THAN",
                "kpiValue": {
                    "floatVal": 100
                },
                "kpiId": {
                    "kpi_id": {
                        "uuid": "2"
                    }
                }
            }
        ]
    }
}
\end{lstlisting}

\section{make file abilene}
\begin{lstlisting}[language=Python]
start: NGSDN_TOPO_PY := 4switch2path.py
start: _start

start-8: NGSDN_TOPO_PY := 8switch3path.py
start-8: _start

start-ab: NGSDN_TOPO_PY := abilene.py
start-ab: _start

start-v4: NGSDN_TOPO_PY := topo-v4.py
start-v4: _start
\end{lstlisting}

\section{File addService con aggiunta capacità}
\label{cap:polcap}
\begin{lstlisting}[language=json]
{
    "serviceId": {
        "context_id": {
            "context_uuid": {
                "uuid": "43813baf-195e-5da6-af20-b3d0922e71a7"
            }
        },
        "service_uuid": {
            "uuid": "be1fbc16-fad4-5832-9a32-754a7d134b81"
        }
    },
    "policyRuleBasic": {
        "priority": 0,
        "policyRuleId": {
            "uuid": {
                "uuid": "1"
            }
        },
        "booleanOperator": "POLICYRULE_CONDITION_BOOLEAN_OR",
        "policyRuleState": {
            "policyRuleStateMessage": ""
        },
        "actionList": [
            {
                "action": "POLICY_RULE_ACTION_RECALCULATE_PATH",
                "action_config": [
                    {
                        "action_key": "",
                        "action_value": ""
                    }
                ]
            }
        ],
        "conditionList": [
            {
                "numericalOperator": "POLICYRULE_CONDITION_NUMERICAL_GREATER_THAN",
                "kpiValue": {
                    "floatVal": 5
                },
                "kpiId": {
                    "kpi_id": {
                        "uuid": "1"
                    }
                }
            },
            {
                "numericalOperator": "POLICYRULE_CONDITION_NUMERICAL_GREATER_THAN",
                "kpiValue": {
                    "floatVal": 100
                },
                "kpiId": {
                    "kpi_id": {
                        "uuid": "2"
                    }
                }
            },
            {
                "numericalOperator": "POLICYRULE_CONDITION_NUMERICAL_LESS_THAN",
                "kpiValue": {
                    "floatVal": 0.002
                },
                "kpiId": {
                    "kpi_id": {
                        "uuid": "3"
                    }
                }
            }
        ]
    }
}
\end{lstlisting}

\section{implementazione probe con aggiunta capacità}
\subsection{agent.py}
\begin{lstlisting}[language=Python]
import os, threading, time, socket
from common.Settings import get_setting
from common.proto.context_pb2 import Empty, Timestamp
from common.proto.monitoring_pb2 import KpiDescriptor, Kpi, KpiId, KpiValue
from common.proto.kpi_sample_types_pb2 import KpiSampleType
from monitoring.client.MonitoringClient import MonitoringClient
from context.client.ContextClient import ContextClient

# ----- If you want to use .env file
#from dotenv import load_dotenv
#load_dotenv()
#def get_setting(key):
#    return os.getenv(key)


#### gRPC Clients
monitoring_client = MonitoringClient(get_setting('MONITORINGSERVICE_SERVICE_HOST'), get_setting('MONITORINGSERVICE_SERVICE_PORT_GRPC'))
context_client = ContextClient(get_setting('CONTEXTSERVICE_SERVICE_HOST'), get_setting('CONTEXTSERVICE_SERVICE_PORT_GRPC'))

### Locks and common variables
enabled_lock = threading.Lock()
kpi_id_lock = threading.Lock()
kpi_id_loss = KpiId()
kpi_id_latency = KpiId()
kpi_id_capacity = KpiId()
enabled = False

### Define the path to the Unix socket
socket_path = "/home/teraflow/ngsdn-tutorial/tmp/tfsping"
if os.path.exists(socket_path):
    os.remove(socket_path)

def thread_context_func():
    global kpi_id_loss
    global kpi_id_latency
    global kpi_id_capacity
    global enabled
    while True:
        # Listen to ContextService/GetServiceEvents stream 
        events = context_client.GetServiceEvents(Empty())
        for event in events:
            event_service = event.service_id
            event_service_uuid = event_service.service_uuid.uuid
            event_type = event.event.event_type
            if event_type == 1:
                print(f"stream: New CREATE event:\n{event_service}")
                #loss
                kpi_descriptor_loss = KpiDescriptor(
                        kpi_description = f"Loss Ratio for service {event_service_uuid}",
                        service_id = event_service,
                        kpi_sample_type = KpiSampleType.KPISAMPLETYPE_UNKNOWN
                        )
                with kpi_id_lock:
                    response_loss = monitoring_client.SetKpi(kpi_descriptor_loss)
                    kpi_id_loss = response_loss
                    print("loss: "+str(kpi_id_loss))
                    
                #latency
                kpi_descriptor_latency = KpiDescriptor(
                        kpi_description = f"Latency for service {event_service_uuid}",
                        service_id = event_service,
                        kpi_sample_type = KpiSampleType.KPISAMPLETYPE_SERVICE_LATENCY_MS
                        )
                with kpi_id_lock:
                    response_latency = monitoring_client.SetKpi(kpi_descriptor_latency)
                    kpi_id_latency = response_latency
                    print("latency: "+str(kpi_id_latency))

                #capacity
                kpi_descriptor_capacity = KpiDescriptor(
                        kpi_description = f"Capacity for service {event_service_uuid}",
                        service_id = event_service,
                        kpi_sample_type = KpiSampleType.KPISAMPLETYPE_LINK_TOTAL_CAPACITY_GBPS
                        )
                with kpi_id_lock:
                    response_capacity = monitoring_client.SetKpi(kpi_descriptor_capacity)
                    kpi_id_capacity = response_capacity
                    print("capacity: "+str(kpi_id_capacity))

                with enabled_lock:
                    enabled = True
            elif event_type == 3:
                print(f"stream: New REMOVE event:\n{event_service}")
                with enabled_lock:
                    enabled = False

def thread_kpi_func():
    global kpi_id_loss
    global kpi_id_latency
    global kpi_id_capacity
    global enabled
    try:
        # Create socket object
        server_socket = socket.socket(socket.AF_UNIX, socket.SOCK_STREAM)

        # Bind the socket to the socket path
        server_socket.bind(socket_path)

        # Listen for incoming connections
        server_socket.listen(1)
        
        while True:
            print("Awaiting for new connection!")

            # Accept incoming connection
            connection, client_address = server_socket.accept()

            # Read data from the connection
            data = connection.recv(1024)

            if data:
                with enabled_lock:
                    if enabled: 
                        data = data.decode()
                        print(f"Received: {data}")
                        latency, loss, capacity = data.split(',')
                        with kpi_id_lock:
                            
                            now = time.time()

                            new_timestamp = Timestamp()
                            new_timestamp.timestamp = now

                            new_value = KpiValue()
                            new_value.floatVal = float(latency)

                            kpi_latency = Kpi (
                                    kpi_id = kpi_id_latency,
                                    timestamp = new_timestamp,
                                    kpi_value = new_value
                                    )
                            print(kpi_latency)
                            response_latency = monitoring_client.IncludeKpi(kpi_latency) 
                        with kpi_id_lock:
                            new_value.floatVal = float(loss)
                            
                            kpi_loss = Kpi(
                                kpi_id=kpi_id_loss,
                                timestamp=new_timestamp,
                                kpi_value=new_value
                            )
                            print(kpi_loss)
                            response_loss = monitoring_client.IncludeKpi(kpi_loss)

                        with kpi_id_lock:
                            new_value.floatVal = float(capacity)
                            
                            kpi_capacity = Kpi(
                                kpi_id=kpi_id_capacity,
                                timestamp=new_timestamp,
                                kpi_value=new_value
                            )
                            print(kpi_capacity)
                            response_capacity = monitoring_client.IncludeKpi(kpi_capacity)

            # Close the connection 
            connection.close()

    
    except Exception as e:
        print(f"Error: {str(e)}")


def main():

    # Start Thread that listens to context events
    thread_context = threading.Thread(target=thread_context_func)
    thread_context.daemon = True
    thread_context.start()

    # Start Thread that listens to socket
    thread_kpi = threading.Thread(target=thread_kpi_func)
    thread_kpi.daemon = True
    thread_kpi.start()

    try:
        while True:
            time.sleep(1)
    except KeyboardInterrupt:
        os.remove(socket_path)
        print("Script terminated.")

if __name__ == "__main__":
    main()
\end{lstlisting}
\subsection{ping2.py}
\begin{lstlisting}[language=Python]
import socket, re, time, subprocess, sys, os

socket_path = "/tmp/tfsping"

#socket_path = "./tmp/sock"
def main():
    hostname = sys.argv[1]
   # server_host = sys.argv[2]
    count = 1
    wait = 5
    packet_sizes = [64, 512, 1024, 2048, 4096]

    total_pings = 0
    successful_pings = 0
    try:
        while True:
            start_time = time.time()

            try:
                # Run the ping command and capture the output
                result = subprocess.check_output(["ping", "-W", str(wait), "-c", str(count), hostname], universal_newlines=True)
                response_time = float(re.findall(r"time=([0-9.]+) ms", result)[0])

#cambia client_host e server_host
                iperf_result = subprocess.check_output(["iperf", "-c", hostname], universal_newlines=True)
                bandwidth_mb = re.findall(r"(\d+\.?\d*)\s+Mbits/sec", iperf_result)[-1]
                #banda in gbps
                bandwidth = float(bandwidth_mb)/1000
           
            except subprocess.CalledProcessError as e:
                # If ping fails return negative response_time
                response_time = -1
                bandwidth = -1

            # Calculate new loss_ratio
            if response_time != -1:
                successful_pings += 1
            total_pings += 1
            moving_loss_ratio = round(((total_pings - successful_pings) / float(total_pings) * 100), 2)

            print("Total pings: {}".format(total_pings))
            print("Successful pings: {}".format(successful_pings))

            print("Packet loss: {}\%".format(moving_loss_ratio))
            print("Latency: {} ms".format(response_time))
            print("Bandwidth: {} Gbps/sec".format(bandwidth))

            data_latency = str(response_time)
            data_loss = str(moving_loss_ratio)
            data_bandwidth = str(bandwidth)
            data = data_latency+","+data_loss+","+data_bandwidth

            # Write results in socket
            try:
                client_socket = socket.socket(socket.AF_UNIX, socket.SOCK_STREAM)
                client_socket.connect(socket_path)
                client_socket.send(data.encode())
                client_socket.close()
            except Exception as e:
                print(e)

            # Calculate the time taken by ping
            execution_time = time.time() - start_time
            # Wait the rest of the time
            wait_time = max(0, 6 - execution_time)
            time.sleep(wait_time)

    except KeyboardInterrupt:
        print("Script terminated.")

if __name__ == "__main__":
    main()
\end{lstlisting}
\section{Rete emulata con la topologia Abilene}
\label{cap:ab}
\begin{lstlisting}[language=Python]
    def config(self, mac=None, ip=None, defaultRoute=None, lo='up', gw=None,
               **_params):
        super(IPv4Host, self).config(mac, ip, defaultRoute, lo, **_params)
        self.cmd('ip -4 addr flush dev %s' % self.defaultIntf())
        self.cmd('ip -6 addr flush dev %s' % self.defaultIntf())
        self.cmd('ip -4 link set up %s' % self.defaultIntf())
        self.cmd('ip -4 addr add %s dev %s' % (ip, self.defaultIntf()))
        if gw:
            self.cmd('ip -4 route add default via %s' % gw)
        # Disable offload
        for attr in ["rx", "tx", "sg"]:
            cmd = "/sbin/ethtool --offload %s %s off" % (
                self.defaultIntf(), attr)
            self.cmd(cmd)

        def updateIP():
            return ip.split('/')[0]

        self.defaultIntf().updateIP = updateIP

class TutorialTopo(Topo):
    """Abilene Topology"""

    def __init__(self, *args, **kwargs):
        Topo.__init__(self, *args, **kwargs)

        # Switches
        NewYork = self.addSwitch('s1', cls=StratumBmv2Switch, cpuport=CPU_PORT)
        Chicago = self.addSwitch('s2', cls=StratumBmv2Switch, cpuport=CPU_PORT)
        WashingtonDC = self.addSwitch('s3', cls=StratumBmv2Switch, cpuport=CPU_PORT)
        Seattle = self.addSwitch('s4', cls=StratumBmv2Switch, cpuport=CPU_PORT)
        Sunnyvale = self.addSwitch('s5', cls=StratumBmv2Switch, cpuport=CPU_PORT)
        LosAngeles = self.addSwitch('s6', cls=StratumBmv2Switch, cpuport=CPU_PORT)
        Denver = self.addSwitch('s7', cls=StratumBmv2Switch, cpuport=CPU_PORT)
        KansasCity = self.addSwitch('s8', cls=StratumBmv2Switch, cpuport=CPU_PORT)
        Houston = self.addSwitch('s9', cls=StratumBmv2Switch, cpuport=CPU_PORT)
        Atlanta = self.addSwitch('s10', cls=StratumBmv2Switch, cpuport=CPU_PORT)
        Indianapolis = self.addSwitch('s11', cls=StratumBmv2Switch, cpuport=CPU_PORT)


        # ... and now hosts
        NewYork_host = self.addHost( 'h0',cls=IPv4Host, mac="aa:bb:cc:dd:ee:11",
                            ip='10.0.0.1/24', gw='10.0.0.100' )
        Chicago_host = self.addHost( 'h1', cls=IPv4Host, mac="aa:bb:cc:dd:ee:22",
                            ip='10.0.0.2/24', gw='10.0.0.100' )
        WashingtonDC_host = self.addHost( 'h2',cls=IPv4Host, mac="aa:bb:cc:dd:ee:33",
                            ip='10.0.0.3/24', gw='10.0.0.100' )
        Seattle_host = self.addHost( 'h3',cls=IPv4Host, mac="aa:bb:cc:dd:ee:44",
                            ip='10.0.0.4/24', gw='10.0.0.100' )
        Sunnyvale_host = self.addHost( 'h4',cls=IPv4Host, mac="aa:bb:cc:dd:ee:55",
                            ip='10.0.0.5/24', gw='10.0.0.100' )
        LosAngeles_host = self.addHost( 'h5',cls=IPv4Host, mac="aa:bb:cc:dd:ee:66",
                            ip='10.0.0.6/24', gw='10.0.0.100' )
        Denver_host = self.addHost( 'h6', cls=IPv4Host, mac="aa:bb:cc:dd:ee:77",
                            ip='10.0.0.7/24', gw='10.0.0.100' )
        KansasCity_host = self.addHost( 'h7',cls=IPv4Host, mac="aa:bb:cc:dd:ee:88",
                            ip='10.0.0.8/24', gw='10.0.0.100' )
        Houston_host = self.addHost( 'h8',cls=IPv4Host, mac="aa:bb:cc:dd:ee:99",
                            ip='10.0.0.9/24', gw='10.0.0.100' )
        Atlanta_host = self.addHost( 'h9',cls=IPv4Host, mac="aa:bb:cc:dd:ee:10",
                            ip='10.0.0.10/24', gw='10.0.0.100' )
        Indianapolis_host = self.addHost( 'h10',cls=IPv4Host, mac="aa:bb:cc:dd:ee:11",
                            ip='10.0.0.11/24', gw='10.0.0.100' )

        # add edges between switches
        self.addLink( NewYork , Chicago, bw=30, delay='0.806374975652ms')
        self.addLink( NewYork , WashingtonDC, bw=30, delay='0.605826192092ms')
        self.addLink( Chicago , Indianapolis, bw=30, delay='1.34462717203ms')
        self.addLink( WashingtonDC , Atlanta, bw=30, delay='0.557636936322ms')
        self.addLink( Seattle , Sunnyvale, bw=30, delay='1.28837123738ms')
        self.addLink( Seattle , Denver, bw=30, delay='1.11169346865ms')
        self.addLink( Sunnyvale , LosAngeles, bw=30, delay='0.590813628707ms')
        self.addLink( Sunnyvale , Denver, bw=30, delay='0.997327682281ms')
        self.addLink( LosAngeles , Houston, bw=30, delay='1.20160833263ms')
        self.addLink( Denver , KansasCity, bw=30, delay='0.223328790403ms')
        self.addLink( KansasCity , Houston, bw=30, delay='1.71325092726ms')
        self.addLink( KansasCity , Indianapolis, bw=30, delay='0.240899959477ms')
        self.addLink( Houston , Atlanta, bw=30, delay='1.34344500256ms')
        self.addLink( Atlanta , Indianapolis, bw=30, delay='0.544962634977ms')


        # add edges between switch and corresponding host
        self.addLink( NewYork , NewYork_host )
        self.addLink( Chicago , Chicago_host )
        self.addLink( WashingtonDC , WashingtonDC_host )
        self.addLink( Seattle , Seattle_host )
        self.addLink( Sunnyvale , Sunnyvale_host )
        self.addLink( LosAngeles , LosAngeles_host )
        self.addLink( Denver , Denver_host )
        self.addLink( KansasCity , KansasCity_host )
        self.addLink( Houston , Houston_host )
        self.addLink( Atlanta , Atlanta_host )
        self.addLink( Indianapolis , Indianapolis_host )

def main():
    net = Mininet(topo=TutorialTopo(), controller=None)
    net.start()
    
    #get hosts new york kansascity
    NewYork_host = net.hosts[0]
    NewYork_host.setARP('10.0.0.8', 'aa:bb:cc:dd:ee:88')
    KansasCity_host = net.hosts[1]
    KansasCity_host.setARP('10.0.0.1', 'aa:bb:cc:dd:ee:11')
   
    
    CLI(net)
    net.stop()
    print '#' * 80
    print 'ATTENTION: Mininet was stopped! Perhaps accidentally?'
    print 'No worries, it will restart automatically in a few seconds...'
    print 'To access again the Mininet CLI, use `make mn-cli`'
    print 'To detach from the CLI (without stopping), press Ctrl-D'
    print 'To permanently quit Mininet, use `make stop`'
    print '#' * 80


if __name__ == "__main__":
    parser = argparse.ArgumentParser(
        description='Mininet topology script for 2x2 fabric with stratum_bmv2 and IPv4 hosts')
    args = parser.parse_args()
    setLogLevel('info')

    main()
\end{lstlisting}