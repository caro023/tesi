\chapter{ONOS}
Open Newtwork Operating System (ONOS) \cite{ONOS} è uno dei controller SDN più noti. 
E' un progetto nato dalla Open Networking Foundation (ONF) \cite{ONF} al fine di soddisfare le esigenze degli operatori per poter costruire reali soluzioni SDN/NFV.
I principali obiettivi sono quelli di introdurre modularità del codice, configurabilità, separazione di interessi e agnosticismo dei protocolli.
\\Per adattarsi alle esigenze degli utenti è necessario poter sviluppare una piattaforma applicativa modulare ed estendibile.
Per questo motivo la base dell'architettura di ONOS è costituita da una piattaforma di applicazioni distributite collocata sopra OSGi \cite{osgi} e Apache Karaf \cite{Apache}. 
Queste applicazioni offrono delle funzionalità di base e sostegno al livello superiore il quale fornisce una serie di controlli di rete e astrazioni di configurazione necessarie per il corretto funzionamento del controller.
Per estendere le funzionalità a seconda delle esigenze sono invece necessarie delle applicazioni ONOS aggiuntive che si comportano come una estenzione di quelle già presenti. 
Ognuna di esse è gestita da un singolo sottosistema che all'interno del controller è rappresentato da un modulo.
I moduli attualmente installabili che si possono incorporare a quelli inizialmente offerti dal sistema sono più di 100.
Tutti i servizi principali sono scritti in Java come bundles all'interno del Karaf OSGi container così da permettere l'installazione e l'esecuzione dinamicamente.
\\ONOS supporta diverse API northbound tra cui:
\begin{itemize}
\item GUI: offre un'interfaccia grafica per interagire con l'utente
\item REST API: facilita l'integrazione con sistemi di orchestrazione e altri controller 
\item gRPC: per un'interazione ad alte prestazioni tra applicazioni e altre entità o protocolli della piattaforma
\end{itemize}
Per quanto riguarda le API southbound supportante fornisce diversi adattatori che rendono il sistema indipendente dai vari protocolli.
\\ONOS è sviluppato come un sistema simmetrico distribuito in cui ogni entità, dal punto di vista software, è indentica alle altre. 
In caso di guasto di una componente le altre sono in grado di sostenere il carico di lavoro assicurando disponibilità.
Pur essendo fisicamente disaggregato offre una visione logicamente centralizzata al fine di fornire l'accesso di ogni informazione alle applicazioni.
\\Per far fronte ai cambiamenti del carico di lavoro o dell'ambiente è dinamicamente scalabile, ossia offre una replica virtualmente illimitata della capacità del piano di controllo.

\begin{figure}[h]
    \centering
   \includegraphics[width=1\textwidth]{archonos.png}
    \caption{Architettura di ONOS}
    \label{fig:img3}
\end{figure}



%altra avaibility provided da core distrubuiti (sonos) 
%variety of devices strong abstraction
%OSGi is a component system for Java that allows modules to be installed and run dynamically in a single JVM. 
%Since ONOS runs in the JVM, it can run on several underlying OS platforms. 

%designed come modular e estensible entity
%develop as a distributed system. is built as a symmetric system dove ogni nodo è software-wise identical
%ogni componente può fare le stesse cose, è fisicamente distribuito, tutte le informazioni sono reperibili
%è dynamicly scalable 
%tutti i messaggi is provided on top of one single messagin substrate cosicchè è più fscile da configurare
%Semplice aggiungere o configurare device e servizi con model based dynamic configurarion.
%Da controllo real-time per dataplane nativi SDN device con OpenFlow o P4 support